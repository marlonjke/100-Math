\chapter{Extremos}
\section{O Problema de Steiner Envolvendo o Número de Euler}
\textit{Em qual valor de $x \geq 0$, a expressão $\sqrt[x]x$ atinjirá seu máximo?}\\
Jacob Steiner publicou este problema no \textit{Crelle's Journal}, vol. XL e em seu livro intitulado \textit{Works}, vol.2, página 243.
Solução:
De acordo com a inequação de funções exponenciais (Número de Euler),
$$e^{(x-e)/e} \geq 1 + \frac{x - e}{e}$$,
onde a igualdade vale somente quando $x=e$. A Inequação é simplificada para
$$e^{x/e}*\frac{1}{e} \geq \frac{x}{e} \textrm{ ou para }  e^{x/e} \geq x.$$
 Extraindo-se a x-ésima raiz quadrada obtém-se
$$\sqrt[e]{e} \geq \sqrt[x]{x}.$$
Conclusão:
\textit{O número  de Euler $e$ é o número que produz o valor máximo para a expressão $\sqrt[x]{x}$ para $x \geq 0$.}

\section{O Problema da Elipse de Steiner}
\textit{De todas as elipses que podem ser circunscritas sobre (inscritas em) um triangulo dado, qual terá a menor (maior) área?}
Dans le plan, la question de polygones d'aire maximum ou minimum inscrits ou circonscrits à une elipse ne présente aucune difficulté. Ill suffit de projeter l'elipse
